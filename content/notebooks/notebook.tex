
% Default to the notebook output style

    


% Inherit from the specified cell style.




    
\documentclass[11pt]{article}

    
    
    \usepackage[T1]{fontenc}
    % Nicer default font (+ math font) than Computer Modern for most use cases
    \usepackage{mathpazo}

    % Basic figure setup, for now with no caption control since it's done
    % automatically by Pandoc (which extracts ![](path) syntax from Markdown).
    \usepackage{graphicx}
    % We will generate all images so they have a width \maxwidth. This means
    % that they will get their normal width if they fit onto the page, but
    % are scaled down if they would overflow the margins.
    \makeatletter
    \def\maxwidth{\ifdim\Gin@nat@width>\linewidth\linewidth
    \else\Gin@nat@width\fi}
    \makeatother
    \let\Oldincludegraphics\includegraphics
    % Set max figure width to be 80% of text width, for now hardcoded.
    \renewcommand{\includegraphics}[1]{\Oldincludegraphics[width=.8\maxwidth]{#1}}
    % Ensure that by default, figures have no caption (until we provide a
    % proper Figure object with a Caption API and a way to capture that
    % in the conversion process - todo).
    \usepackage{caption}
    \DeclareCaptionLabelFormat{nolabel}{}
    \captionsetup{labelformat=nolabel}

    \usepackage{adjustbox} % Used to constrain images to a maximum size 
    \usepackage{xcolor} % Allow colors to be defined
    \usepackage{enumerate} % Needed for markdown enumerations to work
    \usepackage{geometry} % Used to adjust the document margins
    \usepackage{amsmath} % Equations
    \usepackage{amssymb} % Equations
    \usepackage{textcomp} % defines textquotesingle
    % Hack from http://tex.stackexchange.com/a/47451/13684:
    \AtBeginDocument{%
        \def\PYZsq{\textquotesingle}% Upright quotes in Pygmentized code
    }
    \usepackage{upquote} % Upright quotes for verbatim code
    \usepackage{eurosym} % defines \euro
    \usepackage[mathletters]{ucs} % Extended unicode (utf-8) support
    \usepackage[utf8x]{inputenc} % Allow utf-8 characters in the tex document
    \usepackage{fancyvrb} % verbatim replacement that allows latex
    \usepackage{grffile} % extends the file name processing of package graphics 
                         % to support a larger range 
    % The hyperref package gives us a pdf with properly built
    % internal navigation ('pdf bookmarks' for the table of contents,
    % internal cross-reference links, web links for URLs, etc.)
    \usepackage{hyperref}
    \usepackage{longtable} % longtable support required by pandoc >1.10
    \usepackage{booktabs}  % table support for pandoc > 1.12.2
    \usepackage[inline]{enumitem} % IRkernel/repr support (it uses the enumerate* environment)
    \usepackage[normalem]{ulem} % ulem is needed to support strikethroughs (\sout)
                                % normalem makes italics be italics, not underlines
    

    
    
    % Colors for the hyperref package
    \definecolor{urlcolor}{rgb}{0,.145,.698}
    \definecolor{linkcolor}{rgb}{.71,0.21,0.01}
    \definecolor{citecolor}{rgb}{.12,.54,.11}

    % ANSI colors
    \definecolor{ansi-black}{HTML}{3E424D}
    \definecolor{ansi-black-intense}{HTML}{282C36}
    \definecolor{ansi-red}{HTML}{E75C58}
    \definecolor{ansi-red-intense}{HTML}{B22B31}
    \definecolor{ansi-green}{HTML}{00A250}
    \definecolor{ansi-green-intense}{HTML}{007427}
    \definecolor{ansi-yellow}{HTML}{DDB62B}
    \definecolor{ansi-yellow-intense}{HTML}{B27D12}
    \definecolor{ansi-blue}{HTML}{208FFB}
    \definecolor{ansi-blue-intense}{HTML}{0065CA}
    \definecolor{ansi-magenta}{HTML}{D160C4}
    \definecolor{ansi-magenta-intense}{HTML}{A03196}
    \definecolor{ansi-cyan}{HTML}{60C6C8}
    \definecolor{ansi-cyan-intense}{HTML}{258F8F}
    \definecolor{ansi-white}{HTML}{C5C1B4}
    \definecolor{ansi-white-intense}{HTML}{A1A6B2}

    % commands and environments needed by pandoc snippets
    % extracted from the output of `pandoc -s`
    \providecommand{\tightlist}{%
      \setlength{\itemsep}{0pt}\setlength{\parskip}{0pt}}
    \DefineVerbatimEnvironment{Highlighting}{Verbatim}{commandchars=\\\{\}}
    % Add ',fontsize=\small' for more characters per line
    \newenvironment{Shaded}{}{}
    \newcommand{\KeywordTok}[1]{\textcolor[rgb]{0.00,0.44,0.13}{\textbf{{#1}}}}
    \newcommand{\DataTypeTok}[1]{\textcolor[rgb]{0.56,0.13,0.00}{{#1}}}
    \newcommand{\DecValTok}[1]{\textcolor[rgb]{0.25,0.63,0.44}{{#1}}}
    \newcommand{\BaseNTok}[1]{\textcolor[rgb]{0.25,0.63,0.44}{{#1}}}
    \newcommand{\FloatTok}[1]{\textcolor[rgb]{0.25,0.63,0.44}{{#1}}}
    \newcommand{\CharTok}[1]{\textcolor[rgb]{0.25,0.44,0.63}{{#1}}}
    \newcommand{\StringTok}[1]{\textcolor[rgb]{0.25,0.44,0.63}{{#1}}}
    \newcommand{\CommentTok}[1]{\textcolor[rgb]{0.38,0.63,0.69}{\textit{{#1}}}}
    \newcommand{\OtherTok}[1]{\textcolor[rgb]{0.00,0.44,0.13}{{#1}}}
    \newcommand{\AlertTok}[1]{\textcolor[rgb]{1.00,0.00,0.00}{\textbf{{#1}}}}
    \newcommand{\FunctionTok}[1]{\textcolor[rgb]{0.02,0.16,0.49}{{#1}}}
    \newcommand{\RegionMarkerTok}[1]{{#1}}
    \newcommand{\ErrorTok}[1]{\textcolor[rgb]{1.00,0.00,0.00}{\textbf{{#1}}}}
    \newcommand{\NormalTok}[1]{{#1}}
    
    % Additional commands for more recent versions of Pandoc
    \newcommand{\ConstantTok}[1]{\textcolor[rgb]{0.53,0.00,0.00}{{#1}}}
    \newcommand{\SpecialCharTok}[1]{\textcolor[rgb]{0.25,0.44,0.63}{{#1}}}
    \newcommand{\VerbatimStringTok}[1]{\textcolor[rgb]{0.25,0.44,0.63}{{#1}}}
    \newcommand{\SpecialStringTok}[1]{\textcolor[rgb]{0.73,0.40,0.53}{{#1}}}
    \newcommand{\ImportTok}[1]{{#1}}
    \newcommand{\DocumentationTok}[1]{\textcolor[rgb]{0.73,0.13,0.13}{\textit{{#1}}}}
    \newcommand{\AnnotationTok}[1]{\textcolor[rgb]{0.38,0.63,0.69}{\textbf{\textit{{#1}}}}}
    \newcommand{\CommentVarTok}[1]{\textcolor[rgb]{0.38,0.63,0.69}{\textbf{\textit{{#1}}}}}
    \newcommand{\VariableTok}[1]{\textcolor[rgb]{0.10,0.09,0.49}{{#1}}}
    \newcommand{\ControlFlowTok}[1]{\textcolor[rgb]{0.00,0.44,0.13}{\textbf{{#1}}}}
    \newcommand{\OperatorTok}[1]{\textcolor[rgb]{0.40,0.40,0.40}{{#1}}}
    \newcommand{\BuiltInTok}[1]{{#1}}
    \newcommand{\ExtensionTok}[1]{{#1}}
    \newcommand{\PreprocessorTok}[1]{\textcolor[rgb]{0.74,0.48,0.00}{{#1}}}
    \newcommand{\AttributeTok}[1]{\textcolor[rgb]{0.49,0.56,0.16}{{#1}}}
    \newcommand{\InformationTok}[1]{\textcolor[rgb]{0.38,0.63,0.69}{\textbf{\textit{{#1}}}}}
    \newcommand{\WarningTok}[1]{\textcolor[rgb]{0.38,0.63,0.69}{\textbf{\textit{{#1}}}}}
    
    
    % Define a nice break command that doesn't care if a line doesn't already
    % exist.
    \def\br{\hspace*{\fill} \\* }
    % Math Jax compatability definitions
    \def\gt{>}
    \def\lt{<}
    % Document parameters
    \title{RSA-Encryption}
    
    
    

    % Pygments definitions
    
\makeatletter
\def\PY@reset{\let\PY@it=\relax \let\PY@bf=\relax%
    \let\PY@ul=\relax \let\PY@tc=\relax%
    \let\PY@bc=\relax \let\PY@ff=\relax}
\def\PY@tok#1{\csname PY@tok@#1\endcsname}
\def\PY@toks#1+{\ifx\relax#1\empty\else%
    \PY@tok{#1}\expandafter\PY@toks\fi}
\def\PY@do#1{\PY@bc{\PY@tc{\PY@ul{%
    \PY@it{\PY@bf{\PY@ff{#1}}}}}}}
\def\PY#1#2{\PY@reset\PY@toks#1+\relax+\PY@do{#2}}

\expandafter\def\csname PY@tok@w\endcsname{\def\PY@tc##1{\textcolor[rgb]{0.73,0.73,0.73}{##1}}}
\expandafter\def\csname PY@tok@c\endcsname{\let\PY@it=\textit\def\PY@tc##1{\textcolor[rgb]{0.25,0.50,0.50}{##1}}}
\expandafter\def\csname PY@tok@cp\endcsname{\def\PY@tc##1{\textcolor[rgb]{0.74,0.48,0.00}{##1}}}
\expandafter\def\csname PY@tok@k\endcsname{\let\PY@bf=\textbf\def\PY@tc##1{\textcolor[rgb]{0.00,0.50,0.00}{##1}}}
\expandafter\def\csname PY@tok@kp\endcsname{\def\PY@tc##1{\textcolor[rgb]{0.00,0.50,0.00}{##1}}}
\expandafter\def\csname PY@tok@kt\endcsname{\def\PY@tc##1{\textcolor[rgb]{0.69,0.00,0.25}{##1}}}
\expandafter\def\csname PY@tok@o\endcsname{\def\PY@tc##1{\textcolor[rgb]{0.40,0.40,0.40}{##1}}}
\expandafter\def\csname PY@tok@ow\endcsname{\let\PY@bf=\textbf\def\PY@tc##1{\textcolor[rgb]{0.67,0.13,1.00}{##1}}}
\expandafter\def\csname PY@tok@nb\endcsname{\def\PY@tc##1{\textcolor[rgb]{0.00,0.50,0.00}{##1}}}
\expandafter\def\csname PY@tok@nf\endcsname{\def\PY@tc##1{\textcolor[rgb]{0.00,0.00,1.00}{##1}}}
\expandafter\def\csname PY@tok@nc\endcsname{\let\PY@bf=\textbf\def\PY@tc##1{\textcolor[rgb]{0.00,0.00,1.00}{##1}}}
\expandafter\def\csname PY@tok@nn\endcsname{\let\PY@bf=\textbf\def\PY@tc##1{\textcolor[rgb]{0.00,0.00,1.00}{##1}}}
\expandafter\def\csname PY@tok@ne\endcsname{\let\PY@bf=\textbf\def\PY@tc##1{\textcolor[rgb]{0.82,0.25,0.23}{##1}}}
\expandafter\def\csname PY@tok@nv\endcsname{\def\PY@tc##1{\textcolor[rgb]{0.10,0.09,0.49}{##1}}}
\expandafter\def\csname PY@tok@no\endcsname{\def\PY@tc##1{\textcolor[rgb]{0.53,0.00,0.00}{##1}}}
\expandafter\def\csname PY@tok@nl\endcsname{\def\PY@tc##1{\textcolor[rgb]{0.63,0.63,0.00}{##1}}}
\expandafter\def\csname PY@tok@ni\endcsname{\let\PY@bf=\textbf\def\PY@tc##1{\textcolor[rgb]{0.60,0.60,0.60}{##1}}}
\expandafter\def\csname PY@tok@na\endcsname{\def\PY@tc##1{\textcolor[rgb]{0.49,0.56,0.16}{##1}}}
\expandafter\def\csname PY@tok@nt\endcsname{\let\PY@bf=\textbf\def\PY@tc##1{\textcolor[rgb]{0.00,0.50,0.00}{##1}}}
\expandafter\def\csname PY@tok@nd\endcsname{\def\PY@tc##1{\textcolor[rgb]{0.67,0.13,1.00}{##1}}}
\expandafter\def\csname PY@tok@s\endcsname{\def\PY@tc##1{\textcolor[rgb]{0.73,0.13,0.13}{##1}}}
\expandafter\def\csname PY@tok@sd\endcsname{\let\PY@it=\textit\def\PY@tc##1{\textcolor[rgb]{0.73,0.13,0.13}{##1}}}
\expandafter\def\csname PY@tok@si\endcsname{\let\PY@bf=\textbf\def\PY@tc##1{\textcolor[rgb]{0.73,0.40,0.53}{##1}}}
\expandafter\def\csname PY@tok@se\endcsname{\let\PY@bf=\textbf\def\PY@tc##1{\textcolor[rgb]{0.73,0.40,0.13}{##1}}}
\expandafter\def\csname PY@tok@sr\endcsname{\def\PY@tc##1{\textcolor[rgb]{0.73,0.40,0.53}{##1}}}
\expandafter\def\csname PY@tok@ss\endcsname{\def\PY@tc##1{\textcolor[rgb]{0.10,0.09,0.49}{##1}}}
\expandafter\def\csname PY@tok@sx\endcsname{\def\PY@tc##1{\textcolor[rgb]{0.00,0.50,0.00}{##1}}}
\expandafter\def\csname PY@tok@m\endcsname{\def\PY@tc##1{\textcolor[rgb]{0.40,0.40,0.40}{##1}}}
\expandafter\def\csname PY@tok@gh\endcsname{\let\PY@bf=\textbf\def\PY@tc##1{\textcolor[rgb]{0.00,0.00,0.50}{##1}}}
\expandafter\def\csname PY@tok@gu\endcsname{\let\PY@bf=\textbf\def\PY@tc##1{\textcolor[rgb]{0.50,0.00,0.50}{##1}}}
\expandafter\def\csname PY@tok@gd\endcsname{\def\PY@tc##1{\textcolor[rgb]{0.63,0.00,0.00}{##1}}}
\expandafter\def\csname PY@tok@gi\endcsname{\def\PY@tc##1{\textcolor[rgb]{0.00,0.63,0.00}{##1}}}
\expandafter\def\csname PY@tok@gr\endcsname{\def\PY@tc##1{\textcolor[rgb]{1.00,0.00,0.00}{##1}}}
\expandafter\def\csname PY@tok@ge\endcsname{\let\PY@it=\textit}
\expandafter\def\csname PY@tok@gs\endcsname{\let\PY@bf=\textbf}
\expandafter\def\csname PY@tok@gp\endcsname{\let\PY@bf=\textbf\def\PY@tc##1{\textcolor[rgb]{0.00,0.00,0.50}{##1}}}
\expandafter\def\csname PY@tok@go\endcsname{\def\PY@tc##1{\textcolor[rgb]{0.53,0.53,0.53}{##1}}}
\expandafter\def\csname PY@tok@gt\endcsname{\def\PY@tc##1{\textcolor[rgb]{0.00,0.27,0.87}{##1}}}
\expandafter\def\csname PY@tok@err\endcsname{\def\PY@bc##1{\setlength{\fboxsep}{0pt}\fcolorbox[rgb]{1.00,0.00,0.00}{1,1,1}{\strut ##1}}}
\expandafter\def\csname PY@tok@kc\endcsname{\let\PY@bf=\textbf\def\PY@tc##1{\textcolor[rgb]{0.00,0.50,0.00}{##1}}}
\expandafter\def\csname PY@tok@kd\endcsname{\let\PY@bf=\textbf\def\PY@tc##1{\textcolor[rgb]{0.00,0.50,0.00}{##1}}}
\expandafter\def\csname PY@tok@kn\endcsname{\let\PY@bf=\textbf\def\PY@tc##1{\textcolor[rgb]{0.00,0.50,0.00}{##1}}}
\expandafter\def\csname PY@tok@kr\endcsname{\let\PY@bf=\textbf\def\PY@tc##1{\textcolor[rgb]{0.00,0.50,0.00}{##1}}}
\expandafter\def\csname PY@tok@bp\endcsname{\def\PY@tc##1{\textcolor[rgb]{0.00,0.50,0.00}{##1}}}
\expandafter\def\csname PY@tok@fm\endcsname{\def\PY@tc##1{\textcolor[rgb]{0.00,0.00,1.00}{##1}}}
\expandafter\def\csname PY@tok@vc\endcsname{\def\PY@tc##1{\textcolor[rgb]{0.10,0.09,0.49}{##1}}}
\expandafter\def\csname PY@tok@vg\endcsname{\def\PY@tc##1{\textcolor[rgb]{0.10,0.09,0.49}{##1}}}
\expandafter\def\csname PY@tok@vi\endcsname{\def\PY@tc##1{\textcolor[rgb]{0.10,0.09,0.49}{##1}}}
\expandafter\def\csname PY@tok@vm\endcsname{\def\PY@tc##1{\textcolor[rgb]{0.10,0.09,0.49}{##1}}}
\expandafter\def\csname PY@tok@sa\endcsname{\def\PY@tc##1{\textcolor[rgb]{0.73,0.13,0.13}{##1}}}
\expandafter\def\csname PY@tok@sb\endcsname{\def\PY@tc##1{\textcolor[rgb]{0.73,0.13,0.13}{##1}}}
\expandafter\def\csname PY@tok@sc\endcsname{\def\PY@tc##1{\textcolor[rgb]{0.73,0.13,0.13}{##1}}}
\expandafter\def\csname PY@tok@dl\endcsname{\def\PY@tc##1{\textcolor[rgb]{0.73,0.13,0.13}{##1}}}
\expandafter\def\csname PY@tok@s2\endcsname{\def\PY@tc##1{\textcolor[rgb]{0.73,0.13,0.13}{##1}}}
\expandafter\def\csname PY@tok@sh\endcsname{\def\PY@tc##1{\textcolor[rgb]{0.73,0.13,0.13}{##1}}}
\expandafter\def\csname PY@tok@s1\endcsname{\def\PY@tc##1{\textcolor[rgb]{0.73,0.13,0.13}{##1}}}
\expandafter\def\csname PY@tok@mb\endcsname{\def\PY@tc##1{\textcolor[rgb]{0.40,0.40,0.40}{##1}}}
\expandafter\def\csname PY@tok@mf\endcsname{\def\PY@tc##1{\textcolor[rgb]{0.40,0.40,0.40}{##1}}}
\expandafter\def\csname PY@tok@mh\endcsname{\def\PY@tc##1{\textcolor[rgb]{0.40,0.40,0.40}{##1}}}
\expandafter\def\csname PY@tok@mi\endcsname{\def\PY@tc##1{\textcolor[rgb]{0.40,0.40,0.40}{##1}}}
\expandafter\def\csname PY@tok@il\endcsname{\def\PY@tc##1{\textcolor[rgb]{0.40,0.40,0.40}{##1}}}
\expandafter\def\csname PY@tok@mo\endcsname{\def\PY@tc##1{\textcolor[rgb]{0.40,0.40,0.40}{##1}}}
\expandafter\def\csname PY@tok@ch\endcsname{\let\PY@it=\textit\def\PY@tc##1{\textcolor[rgb]{0.25,0.50,0.50}{##1}}}
\expandafter\def\csname PY@tok@cm\endcsname{\let\PY@it=\textit\def\PY@tc##1{\textcolor[rgb]{0.25,0.50,0.50}{##1}}}
\expandafter\def\csname PY@tok@cpf\endcsname{\let\PY@it=\textit\def\PY@tc##1{\textcolor[rgb]{0.25,0.50,0.50}{##1}}}
\expandafter\def\csname PY@tok@c1\endcsname{\let\PY@it=\textit\def\PY@tc##1{\textcolor[rgb]{0.25,0.50,0.50}{##1}}}
\expandafter\def\csname PY@tok@cs\endcsname{\let\PY@it=\textit\def\PY@tc##1{\textcolor[rgb]{0.25,0.50,0.50}{##1}}}

\def\PYZbs{\char`\\}
\def\PYZus{\char`\_}
\def\PYZob{\char`\{}
\def\PYZcb{\char`\}}
\def\PYZca{\char`\^}
\def\PYZam{\char`\&}
\def\PYZlt{\char`\<}
\def\PYZgt{\char`\>}
\def\PYZsh{\char`\#}
\def\PYZpc{\char`\%}
\def\PYZdl{\char`\$}
\def\PYZhy{\char`\-}
\def\PYZsq{\char`\'}
\def\PYZdq{\char`\"}
\def\PYZti{\char`\~}
% for compatibility with earlier versions
\def\PYZat{@}
\def\PYZlb{[}
\def\PYZrb{]}
\makeatother


    % Exact colors from NB
    \definecolor{incolor}{rgb}{0.0, 0.0, 0.5}
    \definecolor{outcolor}{rgb}{0.545, 0.0, 0.0}



    
    % Prevent overflowing lines due to hard-to-break entities
    \sloppy 
    % Setup hyperref package
    \hypersetup{
      breaklinks=true,  % so long urls are correctly broken across lines
      colorlinks=true,
      urlcolor=urlcolor,
      linkcolor=linkcolor,
      citecolor=citecolor,
      }
    % Slightly bigger margins than the latex defaults
    
    \geometry{verbose,tmargin=1in,bmargin=1in,lmargin=1in,rmargin=1in}
    
    

    \begin{document}
    
    
    \maketitle
    
    

    
    \hypertarget{rsa-encryption}{%
\section{RSA Encryption}\label{rsa-encryption}}

With small numbers

Our text to encrypt is just the letter \texttt{B}. We can transform the
letter \texttt{B} into the number 2. From now on, the number 2 will
represent the letter \texttt{B}.

\begin{quote}
We chose the number 2 for the letter B because we consider - A = 1 - B =
2 - C = 3 \ldots{}
\end{quote}

    First some setup for the notebook

    \begin{Verbatim}[commandchars=\\\{\}]
{\color{incolor}In [{\color{incolor}18}]:} \PY{o}{\PYZpc{}}\PY{o}{\PYZpc{}}\PY{n+nx}{javascript}
         \PY{n+nx}{MathJax}\PY{p}{.}\PY{n+nx}{Extension}\PY{p}{[}\PY{l+s+s2}{\PYZdq{}TeX/cancel\PYZdq{}}\PY{p}{]}\PY{o}{=}\PY{p}{\PYZob{}}\PY{n+nx}{version}\PY{o}{:}\PY{l+s+s2}{\PYZdq{}2.4.0\PYZdq{}}\PY{p}{,}\PY{n+nx}{ALLOWED}\PY{o}{:}\PY{p}{\PYZob{}}\PY{n+nx}{color}\PY{o}{:}\PY{l+m+mi}{1}\PY{p}{,}\PY{n+nx}{mathcolor}\PY{o}{:}\PY{l+m+mi}{1}\PY{p}{,}\PY{n+nx}{background}\PY{o}{:}\PY{l+m+mi}{1}\PY{p}{,}\PY{n+nx}{mathbackground}\PY{o}{:}\PY{l+m+mi}{1}\PY{p}{,}\PY{n+nx}{padding}\PY{o}{:}\PY{l+m+mi}{1}\PY{p}{,}\PY{n+nx}{thickness}\PY{o}{:}\PY{l+m+mi}{1}\PY{p}{\PYZcb{}}\PY{p}{\PYZcb{}}\PY{p}{;}\PY{n+nx}{MathJax}\PY{p}{.}\PY{n+nx}{Hub}\PY{p}{.}\PY{n+nx}{Register}\PY{p}{.}\PY{n+nx}{StartupHook}\PY{p}{(}\PY{l+s+s2}{\PYZdq{}TeX Jax Ready\PYZdq{}}\PY{p}{,}\PY{k+kd}{function}\PY{p}{(}\PY{p}{)}\PY{p}{\PYZob{}}\PY{k+kd}{var} \PY{n+nx}{c}\PY{o}{=}\PY{n+nx}{MathJax}\PY{p}{.}\PY{n+nx}{InputJax}\PY{p}{.}\PY{n+nx}{TeX}\PY{p}{,}\PY{n+nx}{a}\PY{o}{=}\PY{n+nx}{MathJax}\PY{p}{.}\PY{n+nx}{ElementJax}\PY{p}{.}\PY{n+nx}{mml}\PY{p}{,}\PY{n+nx}{b}\PY{o}{=}\PY{n+nx}{MathJax}\PY{p}{.}\PY{n+nx}{Extension}\PY{p}{[}\PY{l+s+s2}{\PYZdq{}TeX/cancel\PYZdq{}}\PY{p}{]}\PY{p}{;}\PY{n+nx}{b}\PY{p}{.}\PY{n+nx}{setAttributes}\PY{o}{=}\PY{k+kd}{function}\PY{p}{(}\PY{n+nx}{h}\PY{p}{,}\PY{n+nx}{e}\PY{p}{)}\PY{p}{\PYZob{}}\PY{k}{if}\PY{p}{(}\PY{n+nx}{e}\PY{o}{!==}\PY{l+s+s2}{\PYZdq{}\PYZdq{}}\PY{p}{)}\PY{p}{\PYZob{}}\PY{n+nx}{e}\PY{o}{=}\PY{n+nx}{e}\PY{p}{.}\PY{n+nx}{replace}\PY{p}{(}\PY{l+s+sr}{/ /g}\PY{p}{,}\PY{l+s+s2}{\PYZdq{}\PYZdq{}}\PY{p}{)}\PY{p}{.}\PY{n+nx}{split}\PY{p}{(}\PY{l+s+sr}{/,/}\PY{p}{)}\PY{p}{;}\PY{k}{for}\PY{p}{(}\PY{k+kd}{var} \PY{n+nx}{g}\PY{o}{=}\PY{l+m+mi}{0}\PY{p}{,}\PY{n+nx}{d}\PY{o}{=}\PY{n+nx}{e}\PY{p}{.}\PY{n+nx}{length}\PY{p}{;}\PY{n+nx}{g}\PY{o}{\PYZlt{}}\PY{n+nx}{d}\PY{p}{;}\PY{n+nx}{g}\PY{o}{++}\PY{p}{)}\PY{p}{\PYZob{}}\PY{k+kd}{var} \PY{n+nx}{f}\PY{o}{=}\PY{n+nx}{e}\PY{p}{[}\PY{n+nx}{g}\PY{p}{]}\PY{p}{.}\PY{n+nx}{split}\PY{p}{(}\PY{l+s+sr}{/[:=]/}\PY{p}{)}\PY{p}{;}\PY{k}{if}\PY{p}{(}\PY{n+nx}{b}\PY{p}{.}\PY{n+nx}{ALLOWED}\PY{p}{[}\PY{n+nx}{f}\PY{p}{[}\PY{l+m+mi}{0}\PY{p}{]}\PY{p}{]}\PY{p}{)}\PY{p}{\PYZob{}}\PY{k}{if}\PY{p}{(}\PY{n+nx}{f}\PY{p}{[}\PY{l+m+mi}{1}\PY{p}{]}\PY{o}{===}\PY{l+s+s2}{\PYZdq{}true\PYZdq{}}\PY{p}{)}\PY{p}{\PYZob{}}\PY{n+nx}{f}\PY{p}{[}\PY{l+m+mi}{1}\PY{p}{]}\PY{o}{=}\PY{k+kc}{true}\PY{p}{\PYZcb{}}\PY{k}{if}\PY{p}{(}\PY{n+nx}{f}\PY{p}{[}\PY{l+m+mi}{1}\PY{p}{]}\PY{o}{===}\PY{l+s+s2}{\PYZdq{}false\PYZdq{}}\PY{p}{)}\PY{p}{\PYZob{}}\PY{n+nx}{f}\PY{p}{[}\PY{l+m+mi}{1}\PY{p}{]}\PY{o}{=}\PY{k+kc}{false}\PY{p}{\PYZcb{}}\PY{n+nx}{h}\PY{p}{[}\PY{n+nx}{f}\PY{p}{[}\PY{l+m+mi}{0}\PY{p}{]}\PY{p}{]}\PY{o}{=}\PY{n+nx}{f}\PY{p}{[}\PY{l+m+mi}{1}\PY{p}{]}\PY{p}{\PYZcb{}}\PY{p}{\PYZcb{}}\PY{p}{\PYZcb{}}\PY{k}{return} \PY{n+nx}{h}\PY{p}{\PYZcb{}}\PY{p}{;}\PY{n+nx}{c}\PY{p}{.}\PY{n+nx}{Definitions}\PY{p}{.}\PY{n+nx}{Add}\PY{p}{(}\PY{p}{\PYZob{}}\PY{n+nx}{macros}\PY{o}{:}\PY{p}{\PYZob{}}\PY{n+nx}{cancel}\PY{o}{:}\PY{p}{[}\PY{l+s+s2}{\PYZdq{}Cancel\PYZdq{}}\PY{p}{,}\PY{n+nx}{a}\PY{p}{.}\PY{n+nx}{NOTATION}\PY{p}{.}\PY{n+nx}{UPDIAGONALSTRIKE}\PY{p}{]}\PY{p}{,}\PY{n+nx}{bcancel}\PY{o}{:}\PY{p}{[}\PY{l+s+s2}{\PYZdq{}Cancel\PYZdq{}}\PY{p}{,}\PY{n+nx}{a}\PY{p}{.}\PY{n+nx}{NOTATION}\PY{p}{.}\PY{n+nx}{DOWNDIAGONALSTRIKE}\PY{p}{]}\PY{p}{,}\PY{n+nx}{xcancel}\PY{o}{:}\PY{p}{[}\PY{l+s+s2}{\PYZdq{}Cancel\PYZdq{}}\PY{p}{,}\PY{n+nx}{a}\PY{p}{.}\PY{n+nx}{NOTATION}\PY{p}{.}\PY{n+nx}{UPDIAGONALSTRIKE}\PY{o}{+}\PY{l+s+s2}{\PYZdq{} \PYZdq{}}\PY{o}{+}\PY{n+nx}{a}\PY{p}{.}\PY{n+nx}{NOTATION}\PY{p}{.}\PY{n+nx}{DOWNDIAGONALSTRIKE}\PY{p}{]}\PY{p}{,}\PY{n+nx}{cancelto}\PY{o}{:}\PY{l+s+s2}{\PYZdq{}CancelTo\PYZdq{}}\PY{p}{\PYZcb{}}\PY{p}{\PYZcb{}}\PY{p}{,}\PY{k+kc}{null}\PY{p}{,}\PY{k+kc}{true}\PY{p}{)}\PY{p}{;}\PY{n+nx}{c}\PY{p}{.}\PY{n+nx}{Parse}\PY{p}{.}\PY{n+nx}{Augment}\PY{p}{(}\PY{p}{\PYZob{}}\PY{n+nx}{Cancel}\PY{o}{:}\PY{k+kd}{function}\PY{p}{(}\PY{n+nx}{e}\PY{p}{,}\PY{n+nx}{g}\PY{p}{)}\PY{p}{\PYZob{}}\PY{k+kd}{var} \PY{n+nx}{d}\PY{o}{=}\PY{k}{this}\PY{p}{.}\PY{n+nx}{GetBrackets}\PY{p}{(}\PY{n+nx}{e}\PY{p}{,}\PY{l+s+s2}{\PYZdq{}\PYZdq{}}\PY{p}{)}\PY{p}{,}\PY{n+nx}{f}\PY{o}{=}\PY{k}{this}\PY{p}{.}\PY{n+nx}{ParseArg}\PY{p}{(}\PY{n+nx}{e}\PY{p}{)}\PY{p}{;}\PY{k+kd}{var} \PY{n+nx}{h}\PY{o}{=}\PY{n+nx}{b}\PY{p}{.}\PY{n+nx}{setAttributes}\PY{p}{(}\PY{p}{\PYZob{}}\PY{n+nx}{notation}\PY{o}{:}\PY{n+nx}{g}\PY{p}{\PYZcb{}}\PY{p}{,}\PY{n+nx}{d}\PY{p}{)}\PY{p}{;}\PY{k}{this}\PY{p}{.}\PY{n+nx}{Push}\PY{p}{(}\PY{n+nx}{a}\PY{p}{.}\PY{n+nx}{menclose}\PY{p}{(}\PY{n+nx}{f}\PY{p}{)}\PY{p}{.}\PY{n+nx}{With}\PY{p}{(}\PY{n+nx}{h}\PY{p}{)}\PY{p}{)}\PY{p}{\PYZcb{}}\PY{p}{,}\PY{n+nx}{CancelTo}\PY{o}{:}\PY{k+kd}{function}\PY{p}{(}\PY{n+nx}{e}\PY{p}{,}\PY{n+nx}{g}\PY{p}{)}\PY{p}{\PYZob{}}\PY{k+kd}{var} \PY{n+nx}{i}\PY{o}{=}\PY{k}{this}\PY{p}{.}\PY{n+nx}{ParseArg}\PY{p}{(}\PY{n+nx}{e}\PY{p}{)}\PY{p}{,}\PY{n+nx}{d}\PY{o}{=}\PY{k}{this}\PY{p}{.}\PY{n+nx}{GetBrackets}\PY{p}{(}\PY{n+nx}{e}\PY{p}{,}\PY{l+s+s2}{\PYZdq{}\PYZdq{}}\PY{p}{)}\PY{p}{,}\PY{n+nx}{f}\PY{o}{=}\PY{k}{this}\PY{p}{.}\PY{n+nx}{ParseArg}\PY{p}{(}\PY{n+nx}{e}\PY{p}{)}\PY{p}{;}\PY{k+kd}{var} \PY{n+nx}{h}\PY{o}{=}\PY{n+nx}{b}\PY{p}{.}\PY{n+nx}{setAttributes}\PY{p}{(}\PY{p}{\PYZob{}}\PY{n+nx}{notation}\PY{o}{:}\PY{n+nx}{a}\PY{p}{.}\PY{n+nx}{NOTATION}\PY{p}{.}\PY{n+nx}{UPDIAGONALSTRIKE}\PY{o}{+}\PY{l+s+s2}{\PYZdq{} \PYZdq{}}\PY{o}{+}\PY{n+nx}{a}\PY{p}{.}\PY{n+nx}{NOTATION}\PY{p}{.}\PY{n+nx}{UPDIAGONALARROW}\PY{p}{\PYZcb{}}\PY{p}{,}\PY{n+nx}{d}\PY{p}{)}\PY{p}{;}\PY{n+nx}{i}\PY{o}{=}\PY{n+nx}{a}\PY{p}{.}\PY{n+nx}{mpadded}\PY{p}{(}\PY{n+nx}{i}\PY{p}{)}\PY{p}{.}\PY{n+nx}{With}\PY{p}{(}\PY{p}{\PYZob{}}\PY{n+nx}{depth}\PY{o}{:}\PY{l+s+s2}{\PYZdq{}\PYZhy{}.1em\PYZdq{}}\PY{p}{,}\PY{n+nx}{height}\PY{o}{:}\PY{l+s+s2}{\PYZdq{}+.1em\PYZdq{}}\PY{p}{,}\PY{n+nx}{voffset}\PY{o}{:}\PY{l+s+s2}{\PYZdq{}.1em\PYZdq{}}\PY{p}{\PYZcb{}}\PY{p}{)}\PY{p}{;}\PY{k}{this}\PY{p}{.}\PY{n+nx}{Push}\PY{p}{(}\PY{n+nx}{a}\PY{p}{.}\PY{n+nx}{msup}\PY{p}{(}\PY{n+nx}{a}\PY{p}{.}\PY{n+nx}{menclose}\PY{p}{(}\PY{n+nx}{f}\PY{p}{)}\PY{p}{.}\PY{n+nx}{With}\PY{p}{(}\PY{n+nx}{h}\PY{p}{)}\PY{p}{,}\PY{n+nx}{i}\PY{p}{)}\PY{p}{)}\PY{p}{\PYZcb{}}\PY{p}{\PYZcb{}}\PY{p}{)}\PY{p}{;}\PY{n+nx}{MathJax}\PY{p}{.}\PY{n+nx}{Hub}\PY{p}{.}\PY{n+nx}{Startup}\PY{p}{.}\PY{n+nx}{signal}\PY{p}{.}\PY{n+nx}{Post}\PY{p}{(}\PY{l+s+s2}{\PYZdq{}TeX cancel Ready\PYZdq{}}\PY{p}{)}\PY{p}{\PYZcb{}}\PY{p}{)}\PY{p}{;}\PY{n+nx}{MathJax}\PY{p}{.}\PY{n+nx}{Ajax}\PY{p}{.}\PY{n+nx}{loadComplete}\PY{p}{(}\PY{l+s+s2}{\PYZdq{}[MathJax]/extensions/TeX/cancel.js\PYZdq{}}\PY{p}{)}\PY{p}{;}
\end{Verbatim}


    
    \begin{verbatim}
<IPython.core.display.Javascript object>
    \end{verbatim}

    
    The message that we are encoding is

    \begin{Verbatim}[commandchars=\\\{\}]
{\color{incolor}In [{\color{incolor}2}]:} \PY{n}{message} \PY{o}{=} \PY{l+m+mi}{2}
\end{Verbatim}


    Our encryption key (\texttt{public\_key}) is made of - \texttt{e\ =\ 5}
public (or encryption) exponent - \texttt{n\ =\ 14} modulus

    \begin{Verbatim}[commandchars=\\\{\}]
{\color{incolor}In [{\color{incolor}3}]:} \PY{n}{public\PYZus{}key} \PY{o}{=} \PY{p}{\PYZob{}}\PY{l+s+s1}{\PYZsq{}}\PY{l+s+s1}{e}\PY{l+s+s1}{\PYZsq{}}\PY{p}{:} \PY{l+m+mi}{5}\PY{p}{,} \PY{l+s+s1}{\PYZsq{}}\PY{l+s+s1}{n}\PY{l+s+s1}{\PYZsq{}}\PY{p}{:} \PY{l+m+mi}{14}\PY{p}{\PYZcb{}}
\end{Verbatim}


    The decryption key (\texttt{private\_key}) is made of -
\texttt{d\ =\ 11} private (or decryption) exponent - \texttt{n\ =\ 14}
modulus

    \begin{Verbatim}[commandchars=\\\{\}]
{\color{incolor}In [{\color{incolor}4}]:} \PY{n}{private\PYZus{}key} \PY{o}{=} \PY{p}{\PYZob{}}\PY{l+s+s1}{\PYZsq{}}\PY{l+s+s1}{d}\PY{l+s+s1}{\PYZsq{}}\PY{p}{:} \PY{l+m+mi}{11}\PY{p}{,} \PY{l+s+s1}{\PYZsq{}}\PY{l+s+s1}{n}\PY{l+s+s1}{\PYZsq{}}\PY{p}{:} \PY{l+m+mi}{14}\PY{p}{\PYZcb{}}
\end{Verbatim}


    \hypertarget{encryption}{%
\subsection{Encryption}\label{encryption}}

\begin{itemize}
\tightlist
\item
  \(m\) message to be encrypted
\item
  \(e\) exponent of public key
\item
  \(n\) modulus of public key
\item
  \(c\) encrypted cypher text
\end{itemize}

\textbf{To encrypt some data you need to compute}

\(m^e \mod n = c\)

\textbf{In our example}

\(2^5 mod 14 = 4\)

    \begin{Verbatim}[commandchars=\\\{\}]
{\color{incolor}In [{\color{incolor}5}]:} \PY{c+c1}{\PYZsh{} Compute the encrypted message (cypher text)}
        \PY{n}{encrypted\PYZus{}message} \PY{o}{=} \PY{n}{message} \PY{o}{*}\PY{o}{*} \PY{n}{public\PYZus{}key}\PY{p}{[}\PY{l+s+s1}{\PYZsq{}}\PY{l+s+s1}{e}\PY{l+s+s1}{\PYZsq{}}\PY{p}{]} \PY{o}{\PYZpc{}} \PY{n}{public\PYZus{}key}\PY{p}{[}\PY{l+s+s1}{\PYZsq{}}\PY{l+s+s1}{n}\PY{l+s+s1}{\PYZsq{}}\PY{p}{]}
        
        \PY{c+c1}{\PYZsh{} Display that}
        \PY{n}{encrypted\PYZus{}message}
\end{Verbatim}


\begin{Verbatim}[commandchars=\\\{\}]
{\color{outcolor}Out[{\color{outcolor}5}]:} 4
\end{Verbatim}
            
    \hypertarget{decryption}{%
\subsection{Decryption}\label{decryption}}

\begin{itemize}
\tightlist
\item
  \(c\) cyphertext aka encrypted message (computed above)
\item
  \(d\) exponent of private key
\item
  \(n\) modulus of private key
\item
  \(m\) decrypted message
\end{itemize}

\hypertarget{to-decrypt-some-data-you-need-to-compute}{%
\subsubsection{To decrypt some data you need to
compute}\label{to-decrypt-some-data-you-need-to-compute}}

\(c^d \mod n = m\)

\textbf{In our example}

\(4^{11} mod 14 = 2\)

    \begin{Verbatim}[commandchars=\\\{\}]
{\color{incolor}In [{\color{incolor}6}]:} \PY{c+c1}{\PYZsh{} Compute the decrypted text}
        \PY{n}{decrypted\PYZus{}text} \PY{o}{=} \PY{n}{encrypted\PYZus{}message} \PY{o}{*}\PY{o}{*} \PY{n}{private\PYZus{}key}\PY{p}{[}\PY{l+s+s1}{\PYZsq{}}\PY{l+s+s1}{d}\PY{l+s+s1}{\PYZsq{}}\PY{p}{]} \PY{o}{\PYZpc{}} \PY{n}{private\PYZus{}key}\PY{p}{[}\PY{l+s+s1}{\PYZsq{}}\PY{l+s+s1}{n}\PY{l+s+s1}{\PYZsq{}}\PY{p}{]}
        
        \PY{c+c1}{\PYZsh{} Display decrypted text}
        \PY{n}{decrypted\PYZus{}text}
\end{Verbatim}


\begin{Verbatim}[commandchars=\\\{\}]
{\color{outcolor}Out[{\color{outcolor}6}]:} 2
\end{Verbatim}
            
    \hypertarget{how-to-generate-public-and-private-keys}{%
\subsection{How to generate public and private
keys}\label{how-to-generate-public-and-private-keys}}

Now you understand how to encrypt and decrypt a message if you have the
private and public keys but you also have to learn how to generate the
public and private keys.

We start by picking 2 prime numbers \(p\) and \(q\)

    \begin{Verbatim}[commandchars=\\\{\}]
{\color{incolor}In [{\color{incolor}7}]:} \PY{n}{p} \PY{o}{=} \PY{l+m+mi}{2}
        \PY{n}{q} \PY{o}{=} \PY{l+m+mi}{7}
        
        \PY{n}{p}\PY{p}{,} \PY{n}{q}
\end{Verbatim}


\begin{Verbatim}[commandchars=\\\{\}]
{\color{outcolor}Out[{\color{outcolor}7}]:} (2, 7)
\end{Verbatim}
            
    We compute the \texttt{n} modulus which will be part of both private and
public keys

    \begin{Verbatim}[commandchars=\\\{\}]
{\color{incolor}In [{\color{incolor}8}]:} \PY{n}{n} \PY{o}{=} \PY{n}{p} \PY{o}{*} \PY{n}{q}
        
        \PY{n}{n}
\end{Verbatim}


\begin{Verbatim}[commandchars=\\\{\}]
{\color{outcolor}Out[{\color{outcolor}8}]:} 14
\end{Verbatim}
            
    And we also need to compute \(\phi(n)\)

    \begin{Verbatim}[commandchars=\\\{\}]
{\color{incolor}In [{\color{incolor}9}]:} \PY{n}{phi\PYZus{}n} \PY{o}{=} \PY{p}{(}\PY{n}{p} \PY{o}{\PYZhy{}} \PY{l+m+mi}{1}\PY{p}{)} \PY{o}{*} \PY{p}{(}\PY{n}{q} \PY{o}{\PYZhy{}} \PY{l+m+mi}{1}\PY{p}{)}
        
        
        \PY{n}{phi\PYZus{}n}
\end{Verbatim}


\begin{Verbatim}[commandchars=\\\{\}]
{\color{outcolor}Out[{\color{outcolor}9}]:} 6
\end{Verbatim}
            
    \hypertarget{public-key}{%
\subsubsection{Public key}\label{public-key}}

    We need to pick an \(e\) that fulfills these conditions:

\(e=\left\{  \begin{array}{ll}  1 < e < \phi(n)\\  e \perp n\\  e \perp \phi(n)\\  \end{array}  \right.\)

In this case \(\perp\) means coprime, meaning it does not have any
divisors in common with \(n\) or with \(\phi(n)\).

In other words - \(e\) is greater than \(1\) and lower than \(\phi(n)\)
- \(gcd(e, n) = 1\) - \(gcd(e, \phi(n)) = 1\)

Where \(\gcd\) denotes the greatest common divisor.

    The numbers that are greater than \(1\) and lower than \(\phi(n)\) are

\(2\) \(3\) \(4\) \(5\)

    Out of these numbers, the only number that is coprime with 14 and 6 is

\(\bcancel2\) \(\bcancel3\) \(\bcancel4\) \(5\)

    That means \(e = 5\)

    That means the \textbf{public\_key} is \((e = 5, n = 14)\)

    \hypertarget{private-key}{%
\subsubsection{Private key}\label{private-key}}


    % Add a bibliography block to the postdoc
    
    
    
    \end{document}
